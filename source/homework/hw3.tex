\documentclass[12pt,letterpaper]{report}
%\usepackage{t1enc}
%\usepackage[latin1]{inputenc}
\usepackage[english]{babel}
\usepackage{graphicx}
\pagestyle{empty}
\textwidth6.75in
\oddsidemargin-.25in
\textheight8.5in
\topmargin-.25in

\def\ds{\displaystyle}
\def\ts{\textstyle}
\def\Z{{\bf Z}}
\def\Q{{\bf Q}}
\def\R{{\bf R}}
\def\ps{{\cal P}}
\def\ker{\mbox{ker}}
\def\mod{\mbox{mod }}
\def\implies{\Rightarrow}

\begin{document}

\begin{center}
{\bf Foundations of Mathematics} \\
Homework 3\\
Due Friday, March 7, in class\\
\end{center}

\begin{enumerate}

\item  For each of the following statements, rewrite them either in the form ``if $P$ then $Q$'' or in
the form ``$P$ if and only if $Q$'' without changing their meaning.

\begin{enumerate}
\item  A function is differentiable only if it is continuous.

\item  A function is rational if it is a polynomial.

\item  For matrix $A$ to be invertible, it is necessary and sufficient that $\det(A) \neq 0$.

\item  Whenever people agree with me I feel I must be wrong. (Oscar Wilde)
\end{enumerate}

\item  Recall that the \textbf{converse} of $P\implies Q$ is $Q\implies P$.  Here are some 
questions about statements and their converses.  When you are asked to explain your answer,
give the clearest explanation you can, but don't worry about writing an official proof.  

\begin{enumerate}
\item  Consider the statement ``If $a\in \R$, then $a\in \Q$.''  Is this statement true?  Explain why or why not.

\item  What is the converse of the previous statement?  Is the converse true?  Explain why or why not.

\item  Consider the statement ``If $x\neq 0$, then there exists $y\in\R$ such that $xy = 1$.''  
Is this statement true?  Explain why or why not.

\item  What is the converse of the previous statement?  Is the converse true?  Explain why or why not.
\end{enumerate}

\item  Decide whether or not the following statements are logically equivalent.  
You may use a truth table, or you may
use the rules and definitions from Chapter 2, whichever seems more appropriate for each part,
but be sure to clearly justify your answer.

\begin{enumerate}
\item  $(P\lor Q)\land R$ and $P\lor (Q\land R)$

\item  $P\implies Q$ and $(\sim P)\lor Q$

\item $\sim P$ and $P\implies (Q\land(\sim Q))$
\end{enumerate}

\vfill

\hfill (more on next page)\ 

\newpage



\item  Write each of the following as a sentence in English.   Determine whether the statement is true or false, and explain your answer.  As in the second question, you do not
need to write up formal proofs for these explanations.  

\textit{Remark.} You may retain non-logic symbols like
$\geq$ and $=$, but try to avoid $\in$.  For this exercise, ``there exists a real number~$x$'' is
preferable to ``there exists $x\in\R$. 

\begin{enumerate}
\item  $\forall x\in\R, \exists n\in\Z^+, x^n\geq 0$

\item  $\exists n\in\Z^+, \forall X\in\ps(\Z), |X| < n$

\item  For this part, let $P$ be the set of all prime numbers.

$\forall m\in\Z^+, \exists p\in P, p>m$
\end{enumerate}

\item  Prove that if~$x$ is an odd integer, then~$x^3$ is odd.

\textit{Remark.}  This time, please write a formal direct proof, as in Chapter~4.

\item  Suppose $a,b,c\in\Z$.  Prove that if $a|b$ and $a|c$, then $a|(b+c)$.

\textit{Remark.}  See previous remark.


\end{enumerate}

\bigskip

\begin{center}
\includegraphics[scale=.66]{logiccartoons.eps}
\end{center}

\end{document}