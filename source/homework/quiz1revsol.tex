12pt,letterpaperreport
\usepackage{graphicx}
%\usepackage{t1enc}
%\usepackage[latin1]{inputenc}
\usepackage[english]{babel}
\usepackage{amsmath}
\usepackage{amssymb}
\pagestyle{empty}
\textwidth6.5in
\oddsidemargin-.15in
\textheight9in
\topmargin-.65in

\def\ds{\displaystyle}
\def\ts{\textstyle}
\def\N{{\bf N}}
\def\Z{{\bf Z}}
\def\Q{{\bf Q}}
\def\R{{\bf R}}
\def\ps{{\cal P}}
\def\mod{\mbox{mod }}
\def\emptyset{\varnothing}

\begin{document}

\begin{center}
{\bf Foundations of Mathematics \\
Quiz 1 Review Solutions}\\
\end{center}


\begin{enumerate}

\item Let $A = \{1,2,3\}$, $B = \{\varnothing, 1,\{2\}, \{2,3\}\}$, and $C = \{A,4,5\}$.

\begin{enumerate}
\item  Compute $A\cup C$.

$$A\cup C = \{1,2,3,A,4,5\} = \{1,2,3,4,5,\{1,2,3\}\}$$

\textit{Remark.} $\{1,2,3,A,4,5\}$ on its own is a correct identification of $A\cup C$, and so are 
$\{1,2,3,\{1,2,3\},4,5\}$ and $\{1,2,3,4,5,\{1,2,3\}\}$.


\item  Compute $\ps(A) \cap B$.

$$\ps(A) \cap B = \{\varnothing,\{2\},\{2,3\}\}$$

(Note that you can answer this question \textit{without} computing $\ps(A)$.)

\item  Compute $|\ps (C)|$.

Since $|C| = 3$, $|\ps (C)| = 2^3 = 8$.

\item Is $B\subseteq A$?  Explain your answer.

No, because $\varnothing\in B$ but $\varnothing\notin A$.  

(Note that you can apply the same
argument to $\{2\}$ or  $\{2,3\}$.)

\item Is $A\subseteq C$?  Explain your answer.
 
 No, because $1\in A$ but $1\notin C$.  On the other hand, $A\in C$.
 
 It may (or may not) help to observe that $C = \{\{1,2,3\},4,5\}$.

\end{enumerate}


\item Let $a = 1$, $B = \{1,2\}$, $C = \{\emptyset, \{1\}, \{1,2\}\}$, and $S = \{1,2,3,4,5\}$.

Determine whether each of the following statements is true or false.

\bigskip

\begin{itemize}
\item $a\in S$

true


\item $a\in \ps(S)$

false ($a$ is not a set, so it's not a subset of~$S$.)

\item $B\in S$

false

\item $B\in \ps(S)$ 

true (equivalently, $B\subseteq S$.)

\item $B\subseteq\ps(S)$ 

false (the elements of~$B$ are integers, while the elements of \ps(S) are sets of integers.)

\item $C\in\ps(S)$

false

\item $C\subseteq\ps(S)$

true

\item $C\subseteq\ps(B)$ 

true

\item $\emptyset\in C$ 

true

\item $\emptyset\subseteq C$

true
\end{itemize}

\item Consider the following intervals in $\R$, which will be our universal set for this question.
$$\begin{array}{ccc}
I = [0,2] \ \ \ & K = (1,3]\ \ \  & M = (2,\infty)\end{array}$$

Find the following sets.  If the set is an interval, write it in interval notation.

\begin{enumerate}
\item $I\cup M = [0,\infty)$

\item $I-K = [0,1]$

\item $I\cap K = (1,2]$

\item $I\cap K\cap M = \varnothing$

\item $\overline{M} = (-\infty, 2]$

\item $\overline{I\cup K} = (-\infty, 0)\cup(3,\infty) =
\{x\in\R: x<0 \mbox{ or }x>3\}$

\end{enumerate}

\item The figure below shows  a subset of $\R^2$ of the form 
$$A\times B = \{(x,y): x\in A, y\in B\},$$ where~$A$ and~$B$ are
subsets of~$\R$.  The endpoints of the line segment and the edges of the rectangle are all in the subset.  (The $x$ and $y$ axes are not in the subset.)

\begin{enumerate}
\item  What is~$A$?  What is~$B$?

\medskip

$A = [1,4] = \{x\in\R: 1\leq x\leq 4\}$

\textit{Remark.}  Either of the formulas for~$A$ is valid; you don't need both of them.
But note that $A\neq\{1,4\}$.  Do you see the difference?

$B = [-3,-2]\cup\{-4\} = \{x\in\R: -3\leq x\leq -2\mbox{ or }x = -4\}$
\item  Give the set $(A\times B)\cap(\Z\times\Z)$ by listing its elements between set braces.

$$(A\times B)\cap(\Z\times\Z) = 
\begin{array}{c}\{(1,-2),(1,-3),(1,-4)\ \\
(2,-2),(2,-3),(2,-4)\\
(3,-2),(3,-3),(3,-4)\\
\ (4,-2),(4,-3),(4,-4)\}
\end{array}
$$
\end{enumerate}

\item Consider the sets $$\begin{array}{c}S = \{2,3,5\},\\[\jot]T = \{\varnothing, \{1,2,4,8,16\}\},\\[\jot]
\mbox{and }V = \{5, \{5\}\}.\end{array}$$

Here are five open sentences about a set~$X$.  
For each, list the all the values of~$X$ (chosen from the options $X=S$, $X=T$, and $X=V$) that make the sentence true.  For example, if I gave you the open sentence $5\in X$, the answer would be: $S$, $V$.)

As a reminder, $\Z$ denotes the set of integers.

\begin{enumerate}
\item  $X\in \ps(\Z)$.

\smallskip
$S$
\smallskip

\item  $X\subseteq \ps(\Z)$.

\smallskip
$T$
\smallskip

\item  $X$ has exactly two elements.

\smallskip
$T$, $V$
\smallskip

\item $X\cap\Z = \varnothing$.

\smallskip
$T$
\smallskip

\item $X\cup\Z = \Z$.

\smallskip
$S$
\end{enumerate}


\item 
For the purposes of this question, the math wing is having an open house on Thursday and Friday.  Let~$P$ be the statement ``Kiko is coming to the open house on Thursday'' and let~$Q$ be the statement ``Kiko is coming to the open house on Friday.''

Rewrite each of the following using only the symbols $P$, $Q$, $\land$, $\lor$, $\sim$, $\implies$,
$($, and $)$ without changing its meaning.  

$$\begin{array}{c|c}
\mbox{Symbol} & \mbox{Meaning}\\ \hline
\land&\mbox{and}\\
\lor&\mbox{or}\\
\sim&\mbox{not}\\
\implies&\mbox{implies}\end{array}$$


\begin{enumerate}
\item  Kiko is coming to both days of the open house.

$$P\land Q$$

\item  Kiko is coming to at most one day of the open house.  (Equivalently, Kiko is coming to no more than one day of the open house.)

$$\sim(P\land Q)$$

$$\sim P\ \lor \sim Q$$

$$P\implies \sim Q$$

(Any of these three answers is valid.  There are other possibilities, too.)

\medskip

\item  Kiko is coming to the open house on Friday if she doesn't come on Thursday.

$$\sim P\implies Q$$

(Or, if you want to be too clever by half, $P\lor Q$.)

\medskip

\end{enumerate}

\item  A look at some conditional statements and their converses.

\begin{enumerate}
\item  For this part, $x$ will only take on integer values.

Consider the statement ``If $x$ is even, then $x^2$ is even.''  

Is the statement true or false?

\medskip

It is true.  (Sketch of the proof: Suppose~$x$ is even.  By definition, there is an integer~$n$ for which
$x = 2n$.  It follows that $x^2 = (2n)^2 = 4n^2 = 2(2n^2)$, which is even because $2n^2\in\Z$.)

\bigskip

Give the converse of the statement.  Is the converse true or false?

\medskip

If $x^2$ is even, then $x$ is even.

This is true.  (The proof here is trickier, but we'll be ready for it in a few weeks.)

\bigskip

\item  Consider the statement ``If $a\in\R$, then $a\in\Z$.''

Is the statement true or false?

\medskip

It is false.  For instance, $\pi\in \R$, but $\pi\notin\Z$.

\bigskip

Give the converse of the statement.  Is the converse true or false?

\medskip

If $a\in\Z$, then $a\in\R$

This is true.  This is just another way of saying that $\Z\subseteq\R$.
\end{enumerate}


\end{enumerate}


\end{document}