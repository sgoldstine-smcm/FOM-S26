\documentclass[12pt,letterpaper]{report}
\usepackage{graphicx}
%\usepackage{t1enc}
%\usepackage[latin1]{inputenc}
\usepackage[english]{babel}
\usepackage{amsmath}
\usepackage{amssymb}
\usepackage{titlesec}
%\pagestyle{empty}
%\textwidth6.5in
%\oddsidemargin-.15in
%\textheight9in
%\topmargin-.65in

\def\ds{\displaystyle}
\def\ts{\textstyle}
\def\N{{\bf N}}
\def\Z{{\bf Z}}
\def\Q{{\bf Q}}
\def\R{{\bf R}}
\def\ps{{\cal P}}
\def\mod{\mbox{mod }}
\def\inv{^{-1}}
\def\emptyset{\varnothing}

\title{Axioms of $\R$ and $\Z$}
\titleformat{\section}[display]
{\normalfont\bfseries}
{\thesection.}{0.5em}{\centering}

\titleformat{\subsection}[display]
{\normalfont\itshape}
{\thesection.}{0.5em}{}


\begin{document}

\begin{center}
{\bf Foundations of Mathematics \\
Axioms of~$\R$ and $\Z$}\\
\end{center}

The following statements are all things that you can take for granted when you write proofs in this course.
In practice, you'll probably be fine without consulting this list, but it can be instructive to work out how you use
these properties when you do basic arithmetic and algebra.

\section*{Axioms of~$\R$ that (mostly) also apply to $\Z$}

\subsection*{Operations}

Note that you can always assume the results of basic arithmetic; for instance, you do not need to prove that
$2 + 3 = 5$.  In addition, we have the following axioms.

\begin{itemize}
\item $\R$ is \textbf{closed under addition}.  (Likewise, $\Z$ is closed under addition.)

For all $a,b\in \R$, $a + b\in\R$.  (For all $a,b\in \Z$, $a + b\in\Z$.)

\item $\R$ is \textbf{closed under multiplication}.  (Likewise, $\Z$ is closed under multiplication.)

For all $a,b\in \R$, $a\cdot b\in\R$.  (For all $a,b\in \Z$, $a \cdot b\in\Z$.)

\item While this can be derived from the previous two axioms, it is worth noting that both~$\R$ and~$\Z$ are
also closed under subtraction. 

\item  (Note: you \textbf{cannot} replace~$\R$ with~$\Z$ in this one, but the list would feel incomplete without it.)

For all $a,b\in\R$, if $b\neq 0$, then $a/b\in\R$.

\end{itemize}`

\subsection*{Algebra}

\begin{itemize}
\item  Addition is \textbf{commutative} in~$\R$ (and in~$\Z$).

For all $a,b\in\R$, $a + b = b + a$.

\item Addition is \textbf{associative} in~$\R$ (and in~$\Z$).

For all $a,b,c\in\R$, $(a + b) + c = a + (b + c)$.

\item  Multiplication is commutative in~$\R$ (and in~$\Z$).

For all $a,b\in\R$, $a \cdot b = b \cdot a$.

\item Multiplication is associative in~$\R$ (and in~$\Z$).

For all $a,b,c\in\R$, $(a \cdot b) \cdot c = a \cdot (b \cdot c)$.

\item Multiplication \textbf{distributes} over addition in~$\R$ (and in~$\Z$).

For all $a,b,c\in\R$, $a(b + c) = ab + ac$.

This is also called the \textbf{distributive} property, and it is the basis of all the factoring and expanding
you've done since high school algebra.

\item  Not really an axiom of its own, but a corollary to the standard distributive
property: multiplication also distributes over subtraction in~$\R$ (and in~$\Z$).

For all $a,b,c\in\R$, $a(b - c) = ab - ac$.

\item  This rule, which applies to everything in~$\R$ and~$\Z$,
is the reason why we spend so much time in math classes thinking about how to factor algebraic expressions.

For all $a,b\in\R$, if $a\cdot b = 0$, then $a = 0$ or $b = 0$.
\end{itemize}

While the next cluster of properties is already covered by the blanket statement that you don't have to
``prove'' the results of simple additions and multiplications and by the other axioms 
from \textit{Operations}, I'm including them here because of their significance
in group theory, one of the major components of abstract algebra.

\begin{itemize}

\item  The number~$0$ is the \textbf{additive identity} of~$\R$ (and of~$\Z$).

For all $a\in\R$, $a + 0 = a$.

\item  Every element of~$\R$ has an \textbf{additive inverse} in~$\R$.  (Likewise, every element of~$\Z$
has an additive inverse in~$\Z$.)

For all $a\in \R$, $-a\in \R$, where $a + (-a) = 0$.  (For all $a\in \Z$, $-a\in \Z$, where $a + (-a) = 0$.)

\item  The number~$1$ is the \textbf{multiplicative identity} of~$\R$ (and of~$\Z$).

For all $a\in\R$, $1\cdot a = a$.

\item  Every \textbf{nonzero} element of~$\R$ has a multiplicative inverse in~$\R$.

For all $a\in\R$, if $a\neq 0$, then $a\inv\in\R$ and $a\cdot a\inv = 1$.

\textit{Remarks.}  Recall from exponent rules that $a\inv = 1/a$.  I have my reasons for writing the property using
the exponential notation.

Notice that~$\Z$ is conspicuously absent here.  Do you see why?

\end{itemize}

\subsection*{Inequalities}

Everything in this section  applies to~$\Z$ in exactly the same way it applies to~$\R$, so I will
skip over saying ``and $\Z$'' for everything here.

\begin{itemize}

\item  Suppose $a,b,c\in\R$.  If $a < b$, then $a + c < b + c$.

You can replace $<$ with $\leq$, $>$, or $\geq$.  In fact, if you note that
if $a = b$, then $a + c = b + c$, and also remember that $b > a$ is synonymous with $a < b$, you can derive
each of these three variations from the original statement.

\item  Not really a new axiom, since you can replace $c$ with $-c$ in the axiom above, but it is also
true that for all $a,b,c\in\R$, if $a<b$, then $a-c < b-c$.

\item  Suppose $a,b,c\in\R$ and $a<b$.

If $c > 0$, then $ac < bc$.

If $c < 0$, then $ac > bc$.

You can replace $<$ with $\leq$, and it isn't too difficult to reword this axiom for the assumption that $a > b$.

\item  Suppose $a,b,c\in\R$.

If $a<b$ and $b<c$, then $a < c$.

This is sometimes called the \textbf{transitive} property of~$<$.  The transitive property also applies
to $\leq$, $>$, and $\geq$.

\item \textbf{Trichotomy}:

For any $a,b\in\R$, exactly one of the following three statements is true.

$$a < b$$
$$a = b$$
$$a > b$$

\textit{Remark.}  An alternate form of trichotomy is the statement that for every $a\in\R$, $a$ is
exactly one of the following three things: positive, $0$, or negative.  This version can be combined with
the first axiom in this section to derive the version comparing~$a$ to~$b$.

\end{itemize}

\section*{Axioms particular to~$\Z$}

\subsection*{Mostly just the one axiom}

\begin{itemize}
\item \textbf{The Well Ordering Principle}:

Every non-empty subset of $\Z^+$ has a smallest element.

\textit{Remarks.}  This is somewhat subtler than the other axioms in this list.  To have a better understanding of its significance, 
you might want to think through why the corresponding statement is false for $\R^+$ or for $\Q^+$.

There is a logically equivalent axiom that we'll learn about in a few more chapters.

\end{itemize}

Here is a statement you may also assume in FOM, but it is actually a provable consequence of the Well Ordering Principle.  The proof is
an interesting exercise that you might want to attempt---\textit{after} Chapter 6.

\begin{itemize}
\item  There are no integers between~$0$ and~$1$.

In other words, if $n\in\Z$, then $n\leq 0$ or $n\geq 1$.

\item  A potentially useful corollary of the previous statement:

Suppose $a,b\in\Z$.  If $a > b$, then $a\geq b+1$.

\end{itemize}

Finally, you may use the Division Algorithm (stated in Chapter~1 and again in Chapter~4) without proof.
In Section 1.9, the book outlines how the Division Algorithm can be proven from the Well Ordering Principle.

\begin{itemize}
\item \textbf{Division Algorithm}:

Suppose $a\in\Z$ and $b\in\Z^+$.  Then there exist unique integers~$q$ and~$r$ such that
$a = bq + r$ and $0\leq r < b$.
\end{itemize}


\end{document}
